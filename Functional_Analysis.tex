\documentclass{report}
% Comment the following line to NOT allow the usage of umlauts
\usepackage[utf8]{inputenc}
% Uncomment the following line to allow the usage of graphics (.png, .jpg)
%\usepackage{graphicx}
\usepackage{geometry}
\geometry{left=3cm,right=3cm,top=3cm,bottom=3cm}
\usepackage[all]{xy}
\usepackage{tikz}
\usepackage{amsmath,amsfonts,amssymb,ntheorem}
\usepackage{bigints}
\usepackage{enumitem}
\usepackage{extarrows}
\usepackage{tikz-cd}

\theorembodyfont{\upshape}
\newtheorem{definition}{Definition}[section]
\newtheorem{example}{Example}[section]
\newtheorem{theorem}{Theorem}[section]
\newtheorem{proposition}{Proposition}[section]
\newtheorem{lemma}{Lemma}[section]
\theoremstyle{nonumberplain}
\theoremheaderfont{\itshape}
\theorembodyfont{\normalfont}
\theoremsymbol{\\ \rightline{$\square$}}
\newtheorem{proof}{Proof.}
% Start the document
\begin{document}
\begin{center}
	\textsc{\Huge Functional Analysis}
	~\\
	\vspace{1em}  
	
\end{center}
\vspace{1em} 
\tableofcontents
% Create a new 1st level heading
\chapter{Metric Spaces and Normed Spaces}
\section{Metric Spaces}
\subsection{Basic notions}
\begin{definition}[metric space]
	Given a nonempty set $X$ and a mapping $d:X\times X\to \mathbb{R}$, we say $(X,d)$ is a \emph{metric space} if $d$ satisfies the following conditions for all $x,y\in X$:
	\begin{enumerate}
		\item (identity of indiscernibles): $d(x,y)=0\iff x=y$,
		\item (symmetry): $d(x,y)=d(y,x)$,
		\item (triangle inequality): $d(x,y)\le d(x,z)+d(y,z)$.
	\end{enumerate}	
\end{definition}
\begin{proposition}
	Let $(X,d)$ be a metric space.
	\begin{enumerate}
		\item $d(x,y)\ge 0,\quad\forall x,y\in X$
		\item $|d(x,z)-d(y,z)|\le d(x,y)$ for all $x,y,z\in X$.
		\item $|d(x,y)-d(x',y')|\le d(x,x')+d(y,y')$ for all $x,x',y,y'\in X$.
	\end{enumerate}
\end{proposition}
Metric space can induce a topology on the set $X$ by defining the neighberhood of a point $x$ as 
\[
	B_\epsilon(x)=\left\{y\in X\mid d(x,y)<\epsilon \right\}.
\]
Furthermore, the following proposition holds.

\begin{proposition}[Hausdorffness]
	Metric space is Hausdorff.
\end{proposition}
This suggests us to study sequences in metric spaces since limits of sequences in Hausdorff spaces are unique. 
\begin{definition}[limit of sequence]
	Let $\{x_n\}$ be a sequence in $X$ and $x\in X$. If
	\[
		\lim_{n\to\infty}d(x_n,x)=0,
	\]
	then $x$ is called the \emph{limit} of $\{x_n\}$, denoted as $\lim_{x\to\infty}x_n=x$ or $x_n\to x\;(n\to\infty)$.
\end{definition}

\begin{proposition}[properties of convergent sequences]
	Let $\{x_n\}$ be a convergent sequence in $X$.
	\begin{enumerate}
		\item the limit of $\{x_n\}$ is unique,
		\item $\{x_n\}$ is bounded,
		\item if $x_n\to x$, then any subsequence of $\{x_n\}$ also converges to $x$.
	\end{enumerate}	
\end{proposition}

\begin{proposition}[continuity of metric]
	Given a metric space $(X,d)$, the metric $d:X\times X\to \mathbb{R}$ is continuous, that is,
	\[
		x_n\to x,\; y_n\to y\implies d(x_n,y_n)\to d(x,y).
	\]
\end{proposition}
\begin{definition}[$K$-Lipschitz continuity]
	Let $(X,d_X)$ and $(Y,d_Y)$ be two metric spaces. A map $f:X\to Y$ is called \emph{$K$-Lipschitz continuous} if
	\[
		d_Y(f(x),f(y))\le Kd_X(x,y),\quad\forall x,y\in X.
	\]
\end{definition}
\begin{proposition}
	A $K$-Lipschitz continuous map is continuous.
\end{proposition}
\begin{definition}[short maps]
	Let $(X,d_X)$ and $(Y,d_Y)$ be two metric spaces. A map $f:X\to Y$ is called \emph{short} if it is $1$-Lipschitz continuous, that is,
	\[
		d_Y(f(x),f(y))\le d_X(x,y),\quad\forall x,y\in X.
	\]
\end{definition}
\begin{definition}[isometry]
	Let $(X,d_X)$ and $(Y,d_Y)$ be two metric spaces. A map $f:X\to Y$ is called an \emph{isometry} if
	\[
		d_Y(f(x),f(y))=d_X(x,y),\quad\forall x,y\in X.
	\]	
\end{definition}
\begin{definition}[category of metric spaces]
	Denote the category of metric spaces by $\mathsf{Met}$. The objects of $\mathsf{Met}$ are metric spaces and the morphisms of $\mathsf{Met}$ are short maps.
\end{definition}
\begin{proposition}
	The isomorphisms in $\mathsf{Met}$ are precisely bijective isometries.
\end{proposition}
\begin{proof}
	Let $f:X\to Y$ be an isomorphism in $\mathsf{Met}$. Then $f$ is bijective and short. Since $f^{-1}:Y\to X$ is also short, we have
	\[
		d_X(x,y)=d_X(f^{-1}(f(x)),f^{-1}(f(y)))\le d_Y(f(x),f(y)),\quad\forall x,y\in X.
	\]
	So $f^{-1}$ is an isometry. Conversely, let $f:X\to Y$ be a bijective isometry. Then $f$ is short and $f^{-1}$ is also short. So $f$ is an isomorphism in $\mathsf{Met}$.
\end{proof}
\subsection{Completeness}
\begin{definition}[Cauchy sequences]
	Let $\{x_n\}$ be a sequence in metric space $X$. $\{x_n\}$ is called a \emph{Cauchy sequence} if for all $\epsilon>0$, there exsits $N>0$ such that for all $m,n>N$, $d(x_m,x_n)<\epsilon$.
\end{definition}

\begin{proposition}[properties of Cauchy sequences]
	Cauchy sequences in metric spaces satisfie the following properties:
	\begin{enumerate}
		\item a convergent sequence is a Cauchy sequence,
		\item Cauchy sequences are bounded,
		\item If $\{x_n\}$ is a Cauchy sequence and a subsequence of $\{x_n\}$ converges to $x$, then $\{x_n\}$ also converges to $x$.
	\end{enumerate}	
\end{proposition}

\begin{definition}[complete metric space]
	A metric space is \emph{complete} if and only if all Cauchy sequences are convergent.
\end{definition}

\begin{definition}[pseudometric space]
	A \emph{pseudometric space} $(X, d)$ is a set $X$ together with a non-negative real-valued function $d: X \times X \longrightarrow \mathbb{R}_{\geq 0}$, called a pseudometric, such that for every $x, y, z \in X$,
	\begin{enumerate}
		\item $d(x, x)=0$.
		\item (symmetry): $d(x, y)=d(y, x)$
		\item (triangle inequality): $d(x, z) \leq d(x, y)+d(y, z)$
	\end{enumerate}
\end{definition}

\begin{definition}[metric space induced by a pseudometric]
	Let $(X,d)$ be a pseudometric space. Define an equivalence relation $\sim$ on $X$ as follows: for any $x,y\in X$, $x\sim y$ if and only if $d(x,y)=0$. Denote by $\hat{X}$ the quotient set $X/\sim$. Define a metric $\hat{d}$ on $\hat{X}$ as follows: for any $x,y\in X$, define
	\[
		d^*([x],[y]):=d(x,y).
	\]
\end{definition}
\begin{proof}
	We need to check that $d^*$ is well-defined. Suppose $[x]=[x']$ and $[y]=[y']$, then $d(x,x')=0$ and $d(y,y')=0$. By the triangle inequality, we have
	\begin{align*}
		|d(x,y)-d(x',y')|\le d(x,x')+d(y,y')=0.
	\end{align*}
	So $d(x,y)=d(x',y')$. Thus, $d^*$ is well-defined. 
	Then we can check that $d^*$ is a metric on $\hat{X}$:
	\begin{enumerate}
		\item $d^*([x],[x])=d(x,x)=0$,
		\item $d^*([x],[y])=d(x,y)=d(y,x)=d^*([y],[x])$,
		\item $d^*([x],[z])=d(x,z)\le d(x,y)+d(y,z)=d^*([x],[y])+d^*([y],[z])$.
	\end{enumerate}
\end{proof}

\begin{lemma}
	Let $(X, d)$ be a metric space. If $S$ is a dense subset of $X$ such that every Cauchy sequence in $S$ converges in $X$, then $X$ is complete.
\end{lemma}
\begin{proof}
	Let $\{x_n\}$ be a Cauchy sequence in $X$. Since $S$ is dense in $X$, for any $n\in\mathbb{N}$, there exsits $y_n\in S$ such that $d(x_n,y_n)<\frac{1}{n}$. Note that
	\begin{align*}
		d(y_n,y_m)\le d(y_n,x_n)+d(x_n,x_m)+d(x_m,y_m)<d(x_n,x_m)+\frac{2}{n}+\frac{2}{m}
	\end{align*}
	We see $\{y_n\}$ is a Cauchy sequence in $S$ and converges to some $y\in X$. Since $d(x_n,y_n)\to 0$, we have $x_n\to y$.
\end{proof}

\begin{definition}[completition of metric space]
	We will give both the universal property and explicit construction of the completition of a metric space.\\
	\textbf{$\bullet$ Universal property}\\
	Given a metric space $(X,d)$, the \emph{completition} of $(X,d)$ can be characterized by the following universal property: let $I: \mathsf{ComMet}\to\mathsf{Met}$ be the inclusion functor from the category of complete metric spaces to the category of metric spaces, then the \emph{completition of $(X,d)$} is the initial object of the comma category $\left(\Delta X\downarrow I\right)$, denoted as $\iota:(X,d)\to\left(\hat{X},\hat{d}\right)$. That is, for any $Y\in\mathrm{Ob}(\mathsf{ComMet})$ and any morphism $f:X\to I(Y)$, there exsits a unique morphism $\hat{f}:\hat{X}\to Y$ such that the following diagram commutes:
	\[
		\begin{tikzcd}
			I\left(\hat{X}\right) \arrow[rr, "I\left(\hat{f}\right)", dashed] &                                        & I(Y) \\
																			  & X \arrow[lu, "\iota"] \arrow[ru, "f"'] &     
	\end{tikzcd}
	\]
	\textbf{$\bullet$ Explicit construction}\\
	The existence of the completition of $(X,d)$ can be proved by explicit construction. Denote by $\mathrm{CauSeq}[X]$ the set of all Cauchy sequences in $X$. Define a map
	\begin{align*}
		d_c:\mathrm{CauSeq}[X]\times\mathrm{CauSeq}[X]&\longrightarrow\mathbb{R}_{\ge 0}\\
		\left(\{x_n\},\{y_n\}\right)&\longmapsto\lim_{n\to\infty}d(x_n,y_n).
	\end{align*}
	Since 
	\begin{align*}
		|d(x_n,y_n)-d(x_m,y_m)|\le d(x_n,x_m)+d(y_n,y_m),
	\end{align*}
    we see $d(x_n,y_n)$ is a Cauchy sequence in $\mathbb{R}$, which means the limit $\lim_{n\to\infty}d(x_n,y_n)$ exists. We can check that $d_c$ is a pseudometric on $\mathrm{CauSeq}[X]$:
	\begin{enumerate}
		\item $d_c(\{x_n\},\{x_n\})=\lim_{n\to\infty}d(x_n,x_n)=0$,
		\item $d_c(\{x_n\},\{y_n\})=\lim_{n\to\infty}d(x_n,y_n)=\lim_{n\to\infty}d(y_n,x_n)=d_c(\{y_n\},\{x_n\})$,
		\item $d_c(\{x_n\},\{z_n\})=\lim_{n\to\infty}d(x_n,z_n)\le\lim_{n\to\infty}d(x_n,y_n)+\lim_{n\to\infty}d(y_n,z_n)=d_c(\{x_n\},\{y_n\})+d_c(\{y_n\},\{z_n\})$.
	\end{enumerate}
	The pseudometric $d_c$ induces a metric $\hat{d}:=d_c^*$ on $\hat{X}:=\mathrm{CauSeq}[X]/\sim$
	\begin{align*}
		\hat{d}:\hat{X}\times\hat{X}&\longrightarrow\mathbb{R}_{\ge 0}\\
		\left([\left\{x_n\right\}],[\left\{y_n\right\}]\right)&\longmapsto\lim_{n\to\infty}d(x_n,y_n).
	\end{align*}
	Define 
	\begin{align*}
		\iota:X &\longrightarrow\hat{X}\\
		x&\longmapsto[\left\{x,x,x,\cdots\right\}].
	\end{align*}
	$\iota$ is an isometry because
	\[
		\hat{d}\left(\iota\left(x\right),\iota\left(y\right)\right)=\lim_{k\to\infty}d(x,y)=d(x,y),\quad\forall x,y\in X.
	\]
	\textbf{$\bullet$ Verification of the universal property}\\
	We can verify that $\iota:(X,d)\to\left(\hat{X},\hat{d}\right)$ is the completition of $(X,d)$. 
	\begin{enumerate}[label=(\roman*)]
		\item We can utilize lemma to show $\hat{X}$ is complete. Suppose $\left[\left\{x_n\right\}\right]\in \hat{X}$. Since $\left\{x_n\right\}$ is a Cauchy sequence, there exists a $N>0$ such that for all $n,m>N$, $d(x_m,x_n)<\epsilon$. So whenever $m>N$, we have
	\[
		\hat{d}\left(\iota\left(x_m\right),[\left\{x_n\right\}]\right)=\lim_{n\to\infty}d(x_m,x_n)<\epsilon.
	\]
	Thus we see there exists a sequence $\left\{\iota(x_n)\right\}$ in $\iota(X)$ such that 
	\[
		\lim_{m\to \infty}\hat{d}\left(\iota\left(x_m\right),[\left\{x_n\right\}]\right)\le \epsilon.
	\]
	That implies 
	\[
		\lim_{m\to \infty}\iota\left(x_m\right)= [\left\{x_n\right\}]
	\]
	and accordingly $\iota\left(X\right)$ is dense in $\hat{X}$.\\
	To show $\hat{X}$ is complete, it is sufficient to show every Cauchy sequence in $\iota(X)$ converges in $\hat{X}$. Suppose $\left\{\iota(x_n)\right\}$ is a Cauchy sequence in $\hat{X}$. Given that $\iota$ is an isometry, $\left\{x_n\right\}$ must be a Cauchy sequence in $X$. Hence for the same reason as above, we have
	\[
		\lim_{m\to \infty}\iota\left(x_m\right)= [\left\{x_n\right\}]\in \hat{X}.
	\]
	Therefore, we show that $\hat{X}$ is complete.
	\item Let $Y$ be a complete metric space and $f:X\to I(Y)$ be a morphism in $\mathsf{Met}$. Define a map
	\begin{align*}
		\hat{f}:\hat{X}&\longrightarrow Y\\
		[\left\{x_n\right\}]&\longmapsto\lim_{n\to\infty}f(x_n)
	\end{align*}
	We can check that $\hat{f}$ is well-defined. Suppose $\left\{x_n\right\}\sim\left\{y_n\right\}$, then $\lim_{n\to\infty}d(x_n,y_n)=0$. Since $f$ is a short map, we have
	\[
		d_Y\left(\lim_{n\to\infty}f(x_n),\lim_{n\to\infty}f(y_n)\right)=\lim_{n\to\infty}d_Y(f(x_n),f(y_n))\le \lim_{n\to\infty}d(x_n,y_n)=0,
	\]
	which implies $\hat{f}\left([\left\{x_n\right\}]\right)=\hat{f}\left([\left\{y_n\right\}]\right)$.
	\end{enumerate}
\end{definition}

\section{Normed Spaces}
\subsection{Basic notions}
\begin{definition}[normed space]
	Given a vector space $X$ over $\Bbbk=\mathbb{R}\text{ or }\mathbb{C}$ and a mapping $\Vert\cdot\Vert : X\to\mathbb{R}$, we say
	$\left(X, \Vert\cdot\Vert\right)$ is a \emph{normed vector space} or \emph{normed space} if $\Vert\cdot\Vert$ satisfies the following conditions:
	\begin{enumerate}
		\item positive definiteness: $\forall x\in X,\;\Vert x\Vert\ge 0$ and $\Vert x\Vert=0\iff x=0$,
		\item absolute homogeneity: $\forall x\in X,\;\forall\lambda\in\Bbbk,\;\Vert \lambda x\Vert=|\lambda| \Vert  x\Vert$,
		\item triangle inequality: $\forall x, y\in X,\;\Vert x+ y\Vert\le\Vert x\Vert+\Vert y\Vert$.
	\end{enumerate}	
\end{definition}

\begin{definition}[Banach space]
	A normed space is called a \emph{Banach space} if it is complete.
\end{definition}

\begin{proposition}
	Let $(X,\|\cdot\|)$ be a normed space.
	\begin{enumerate}
		\item $\|\cdot\|:X\to\mathbb{R}$ is continuous, namely
		\[\lim_{n\to\infty}x_n=x\implies\lim_{n\to\infty}\|x_n\|=\|x\|,\]
		\item the addition and multiplication on $X$ is continuous, namely
		\[
			\lim_{n\to\infty}x_n=x,\lim_{n\to\infty}y_n=y,\lim_{n\to\infty}\lambda_n=\lambda\implies\lim_{n\to\infty}x_n+\lambda_ny_n=x+\lambda y.
		\]
	\end{enumerate}	
\end{proposition}

\subsection{$L^p$ spaces}
In general, $L^p\,(1\le p\le\infty)$ spaces are Banach spaces defined on some measure space $(\Omega, \mathcal{F}, \mu)$. For functional analysis, we mainly focus on two types of $L^p$ spaces. 
\subsubsection{$L^p(E)$ spaces}
The first type is defined on some sub measure space $E$ of the Lebesgue measure space $(\mathbb{R}^n, \mathcal{M}, \mu_{\mathcal{M}})$, written as $L^p(E)$ spaces. In this case, the underlying set of $L^p(E)$ consists all measurable functions $f$ such that $\Vert f\Vert_p <\infty$ where $\Vert \cdot\Vert_p:\{f:f\text{ is measurable function on }E\}\to[0,\infty]$ is defined as
\[
	\Vert f\Vert_p:=
	\begin{cases}
		\left(\bigintsss_E |f|^pdx\right)^{\frac{1}{p}},&1\le p< \infty,\\[10pt]
		\inf \left\{M: |f|\le M\text{ a.e.}\right\},&p=\infty.
	\end{cases}	
\]
The norm of $L^p(E)$ is just the restriction of $\Vert \cdot\Vert_p$ to the underlying set of $L^p(E)$. Strictly speaking, measurable functions in $L^p(E)$ should quotient out the equivalent relation $\xlongequal{a.e.}$.
\subsubsection{$\ell^p$ spaces}
The second type is defined on $(\mathbb{N},2^\mathbb{N},\mu_c)$ where $2^\mathbb{N}$ is the power set of $\mathbb{N}$ and $\mu_c$ is the counting measure
\[
\mu_c(A)= 
\begin{cases}
	|A|, & \text { if } A \text { is finite}, \\
	\infty, & \text { if } A \text { is infinite}.
\end{cases}
\]
$L^p(\mathbb{N},\mathcal{P}(\mathbb{N}),\mu_c)$ are often written as $\ell^p$. In this case, the underlying set of $\ell^p$ consists all real number sequences $x=(x_1,x_2,\cdots)$ such that $\Vert x\Vert_p <\infty$ where $\Vert \cdot\Vert_p:\mathbb{R}^\mathbb{N}\to[0,\infty]$ is defined as
\[
	\Vert x\Vert_p:=
	\begin{cases}
		\left(\,\sum\limits_{i=1}^\infty |x_i|^pdx\right)^{\frac{1}{p}},&1\le p< \infty,\\[10pt]
		\;\sup\limits_{i\ge 1}|x_i|,&p=\infty.
	\end{cases}	
\]

\chapter*{Appendix}

\end{document}
